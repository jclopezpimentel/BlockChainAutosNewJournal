\subsection{Blockchain}

Blockchain technology can be abstracted to recall a series of blocks chained in series. In this abstraction, each block of the blockchain represents an object that records a transaction and the chaining is obtained by referencing the previously last block keeping track of the whole history starting from a first block, called \textit{genesis block}. Blockchain is a transaction record keeping platform that tries to attack a series of problems contained in other similar purpose platforms. It achieves its purposes by having the following features \cite{block}.

\begin{itemize}
\item Immutable: It is not possible for a given block to be modified without the agreement of the whole network.
\item Distributed system: Multiple copies of the blockchain exist among its members.
\item No centralized server: Blockchain does not depend on a central authority that dominates the system.            
\end{itemize}
Even though all these features are desired, when blockchain is mentioned as a solution is because of decentralization. Thus, giving control of the system to networks where the main beneficiaries are the users without necessarily having a government figure interested in controlling those exchanges, or where the users prefer to not involve third parties, due to the costs involved or the cumbersomeness of the extra steps required. 

The distribution and immutability of the chain also removes the trust component of a transaction \cite{iot}, i.e. making it a secure way to exchange titles of property between strangers.
