\begin{frontmatter}

%% Title, authors and addresses

%% use the tnoteref command within \title for footnotes;
%% use the tnotetext command for theassociated footnote;
%% use the fnref command within \author or \address for footnotes;
%% use the fntext command for theassociated footnote;
%% use the corref command within \author for corresponding author footnotes;
%% use the cortext command for theassociated footnote;
%% use the ead command for the email address,
%% and the form \ead[url] for the home page:
%% \title{Title\tnoteref{label1}}
%% \tnotetext[label1]{}
%% \author{Name\corref{cor1}\fnref{label2}}
%% \ead{email address}
%% \ead[url]{home page}
%% \fntext[label2]{}
%% \cortext[cor1]{}
%% \address{Address\fnref{label3}}
%% \fntext[label3]{}




%\title{A Distributed Model to Store Vehicle Transaction Records Through Blockchain Platform}
\title{Towards a Distributed Network Model to Store Vehicle Transaction Records Through Blockchain Platform}
%% use optional labels to link authors explicitly to addresses:
%% \author[label1,label2]{}
%% \address[label1]{}
%% \address[label2]{}

\author{Juan-Carlos L\'opez-Pimentel,
        Miguel Alcaraz Rivera,
        %~\IEEEmembership{Fellow,~OSA,},
        Leonardo J. Valdivia,
        and Carolina del Valle Soto% <-this % stops a space
}

\address{}

\begin{abstract}
%% Text of abstract
Blockchain technology, from its emergence until now, has been applied to several projects and 
not just for cryptocurrencies. 
Although this technology has been used for several challenges, to our knowledge, there are no 
records of having applied it to smart properties in vehicles.
In this work, we present a transaction based solution to handle motorized vehicles ownership 
and provenance. 
The solution is based on a distributed model using smart properties and blockchain technologies,  
which offers a secure and useful system where transactions can be requested to handle, 
validate and update the vehicle information. 
The model includes the specification of various client-server security protocols verified formally using AVISPA-SPAN tool.  
The client can have different roles accordingly to the relationship it has with the 
physical property and it can get services and request transactions;
meanwhile the \blockchaincarnetwork  acts as the server entity. 
%Some transaction examples are shown. 
We believe that our proposal can contribute to reduce frauds and increase the trend of 
paperless culture that is strongly being promoted in recent years.

%We present a transaction based solution to handle motorized vehicles ownership 
%and provenance over a distributed model using smart contracts,  
%including the specification of client-server security protocols,
%roles, services, transactions and the \blockchaincarnetwork acting as server.
%Our proposal will reduce frauds and increase the trend of paperless culture.
  %  %Blockchain technology, from its emergence until now, has been applied to several projects and 
%not just for cryptocurrencies. 
%Although this technology has been used for several challenges, to our knowledge, there are no 
%records of having applied it to smart properties in vehicles.
%In this work, we present a transaction based solution to handle motorized vehicles ownership 
%and provenance. 
%The solution is based on a distributed model using smart properties and blockchain technologies,  
%which offers a secure and useful system where transactions can be requested to handle, 
%validate and update the vehicle information. 
%The model includes the specification of various client-server security protocols.  
%The client can have different roles accordingly to the relationship it has with the 
%physical property and it can get services and request transactions;
%meanwhile the \blockchaincarnetwork  acts as the server entity. 
%%Some transaction examples are shown. 
%We believe that our proposal can contribute to reduce frauds and increase the trend of 
%paperless culture that is strongly being promoted in recent years.

We present a transaction based solution to handle motorized vehicles ownership 
and provenance over a distributed model using smart contracts,  
including the specification of client-server security protocols,
roles, services, transactions and the \blockchaincarnetwork acting as server.
Our proposal will reduce frauds and increase the trend of paperless culture.
\end{abstract}

\begin{keyword}
%% keywords here, in the form: keyword \sep keyword
Smart contract 
\sep Smart property
\sep Blockchain
\sep Cloud Computing
\sep Security Protocols
%% PACS codes here, in the form: \PACS code \sep code

%% MSC codes here, in the form: \MSC code \sep code
%% or \MSC[2008] code \sep code (2000 is the default)

\end{keyword}

\end{frontmatter}
