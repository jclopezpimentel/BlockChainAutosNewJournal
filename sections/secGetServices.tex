\section{Get services}
\label{sec:getServices}
Figure~\ref{fig:dfdForAuthServices} illustrates a Data Flow Diagram (DFD) about consuming  
services and set transactions. The Figure shows the scenarios when authentication is or 
not required. Independently of that, the secure channel (described in Section~\ref{sec:secureChannel}) 
is required.

\begin{figure}[hb]
 %\begin{center}
  \centering
    \includegraphics[scale=0.35]{images/lopez3.png}
        \caption{General DFD about consuming services and set transactions}
    \label{fig:dfdForAuthServices}
 % \end{center}
\end{figure}

\begin{table}[htb]
\footnotesize
    \begin{center}
    \caption{Services, require or nor authentication }
    \label{table:servicesPermits}
        \begin{tabular}{l|l|l|l}
			\textbf{Num.}&\textbf{Service}		 &Yes    &Not    \\ \hline
             1			 &Theft status           &       &X      \\ \hline
             2			 &Debt status            &       &X       \\ \hline
             3			 &Basic characteristics  &       &X       \\ \hline
             4			 &Owners                 &X      &       \\ \hline
             5			 &Arrested history       &X      &       \\ \hline
             6			 &Penalties history      &X      &       \\ \hline
             7			 &Tax history            &X      &       \\ \hline
        \end{tabular}
    \end{center}
\end{table}


\subsection{Getting services without authentication}
\label{ssec:getServNoAuth}
Some services must be available for all users and without the need to be 
authenticated 
(Table~\ref{table:servicesPermits} defines what services require or not authentication), 
achieving thus transparency. 
For example, to know:
\begin{itemize}
    \item If a vehicle hold currently a theft or debt report.
    %\item If a vehicle hold currently a debt report.
    \item Some basic characteristics such as \textit{trademark}, \textit{model}, 
        \textit{class}, \textit{version} and \textit{cylinders}.
\end{itemize}


%The network connection starts when a user $C$ contacts to the \blockchaincarnetwork~ 
%through miner $\Server$. 
Table~\ref{table:ProtGetServicesNoAuth} describes how to get services without authentication,
the explanation is as follows:
\begin{itemize}
  \item  The initial knowledge consists in all values learned by the client, $C$, and the 
    \blockchaincarnetwork~ $\Server$ after completed the secure channel stage 
    (described in Section~\ref{sec:secureChannel}).
   \item The request of a service consists in sending an identity $C$, the data of the $QR$ code, and a 
        number $n$ denoting the number of service that is requested to be consumed. 
        The client ciphers  these messages using the secret key $K_{C\Server}$ obtained in
        the secure channel stage. 
        %(see Section~\ref{sec:secureChannel}).
   \item The miner $\Server$ (having a copy of all blockchains within the \blockchaincarnetwork) verifies 
        if such a service can be consumed (according permissions) and then, 
        returns the answer $d$; ciphered using, again, the shared secret key $K_{C\Server}$. Finally,
        $C$ decodes the message and shows the information $d$ to the user.
\end{itemize}


\begin{table}[htb]
\footnotesize
\begin{center}
\caption{Get Service Protocol, authentication is not required}
\label{table:ProtGetServicesNoAuth}
\begin{tabular}{|l|}
\hline
           Initial Knowledge                                                             \\
            $C:C\lnk QR \lnk  K_{C\Server}$                                    \\
            $\Server: \Server\lnk K_{C\Server}$    \\ \hline \hline 
           Get a service                                                                        \\
           1.-$\Client\rightarrow \Server: \fat{C \lnk QR \lnk n}_{K_{C\Server}}$          \\ 
           \hspace{5mm} $d=getService(QR,n)$                                  \\  
           2.-$\Server\rightarrow \Client: \fat{C \lnk d}_{K_{C\Server}}$       \\  \hline \hline
\end{tabular}
\end{center}
\end{table}
\normalsize


\subsection{Get Services with authentication}
\label{ssec:ServAuth}
Sometimes it will be necessary to know who is consuming some services; or maybe the service consumes extra 
computational power. For that, client and miner must be authenticated. 

Transport Layer Security (TLS) allows client/server agents to communicate 
across a hostile network, but eventually, only the server is authenticated. 
Mutual authentication requires a public key infrastructure for clients, which
we achieve by mean a \textit{registration process}. 

\subsubsection{Registration with the Blockchain vehicle wallet}
\label{sec:Registration}

This part consists in to establish a \textit{username} and a \textit{password}, which
will be mapped with an identity. The \textit{username} must be a valid email and the 
\textit{password} will be created by the user.

Table~\ref{table:reqSessKey} summarizes the protocol
in Alice\&Bob notation, its description is as follows: 
\begin{itemize}
  \item Firstly, the user must establish a secure channel with the miner-vehicle-wallet\footnote{
    We will use server-wallet to abbreviate miner-vehicle-wallet.},
    $K_{C\ServerW}$,
    (see Section~\ref{sec:secureChannel}). In this step all exchanged messages are ciphered 
    using the symmetric key $K_{C\ServerW}$, which was obtained during the TLS protocol. 
  \item Secondly, the user must generate his key pair (public and private) through
    his username and password executing the following steps:
    \begin{itemize}
    \item \textbf{First step:} the client sends his username denoted with his email $@$, and  
        a list of personal data $Pd$. 
    \item \textbf{Second step:} the server-wallet sends back the user via an email server $\Client_e$ a 
        data obtained of the client's personal information, $Pd$ and a challenge $X=f(Pd)$. 
    \item \textbf{Third step:} then, the client must generate the final password $P$, but
      only the hash is sent, $Hash(P,X)$.  Sending $Hash(P,X)$, we can
      ensure that $P$ will be only known by $\Client$.
    \item \textbf{Fourth step:} the miner-wallet creates a public and private key for the client 
        ($K_{pub(\Client)}$ and $K_{priv(\Client)}$), which the system maps with the client and 
        the hashed password; builds a block $bp$ containing the public key of the client and broadcasts
        it to the \blockchaincarnetwork to do the mining process. 
    \item \textbf{Fifth step:} After the mining process, the miner-wallet provides to the $\Client$ the 
        public and private key; which will be used to do transactions. 
    \end{itemize}
\end{itemize}
Note that, any user could transact a lot of accounts with the miner-wallet, but each of their 
vehicle-wallet would be empty.

\begin{table}[htb]
\footnotesize
\begin{center}
\caption{Client registration process}
\label{table:reqSessKey}
\begin{tabular}{|l|}
\hline
%{\bf Process}                                                                \\\hline\hline
%Initial Knowledge                                                             \\
%$C:N_{C}\lnk K_{pub(S)}\lnk N'_{C} \lnk K'_{CS}$                              \\
%$\ServerW: K_{pub(S)}\lnk K_{priv(S)}\lnk \ServerW\lnk N_{C}\lnk N_{C'}\lnk 
%                      K'_{CS}$                                                \\ \hline \hline
\textbf{Initial Knowledge}                                                        \\
$\Client:K_{\client\serverW}\lnk [@::Pd]$                                                    \\
$\ServerW: K_{\client\serverW}$                                                            \\ \hline \hline
1.-$\Client\rightarrow \ServerW:\fat{[@::Pd]}_{K_{\client\serverW}}$                     \\
\hspace{5mm} $X=f(Pd) $                           \\ 
2.-$\ServerW\rightarrow \Client_e: \fat{@\lnk Pd \lnk X}_{K_{\client_e\serverW}} $                  \\ 
\hspace{5mm} $\Client$ generates password $P$                          \\  
3.-$\Client\rightarrow \ServerW:\fat{@\lnk Hash(P,X)}_{K_{\client\serverW}}$                 \\  
\hspace{5mm} $\ServerW$ generates a pair key ($K_{pub(\Client)}$ and $K_{priv(\Client)}$)\\
\hspace{5mm} $\ServerW$ generates a block $bp = blockPubKey(K_{pub(\Client)})$\\
4.-$\ServerW\rightarrow [\Server_i .. \Server_n]: \fat{\ServerW \lnk bp}_{K_{\Server_i .. \Server_n}}$  \\  
5.-$\ServerW\rightarrow \Client: \fat{C \lnk K_{pub(\Client)}\lnk K_{priv(\Client)}}_{K_{\client\serverW}}$          \\\\            \hline
\textbf{Security Properties}                        \\
\hspace{5mm}Secrecy of $P$ known by $[\Client]$ \\
\hspace{5mm}Secrecy of $K_{pub(\Client)}$ known by $[\Client,\ServerW]$ \\
\hspace{5mm}Secrecy of $K_{priv(\Client)}$ known by $[\Client,\ServerW]$ \\\hline \hline 
\end{tabular}
\end{center}
\end{table}
\normalsize




The security properties the protocol must provides are: $P$ to be
known only by $C$; $K_{pub(\Client)}$ and $K_{priv(\Client)}$ to be known by $\Server$; 
and obtain strong authentication between the participants.

% We have verified this protocol using The AVISPA Tool Web
% Interface, available via http://www.avispa-project.org/
% finding the following:
% \begin{itemize}
%     \item The protocol provides secrecy.
%     \item We have proved that the protocol provides
%       \emph{authentication} from the client to the server on message 
%       $f(Pd)$ and \emph{weak authentication} from the server to the
%       client on message $Hash(P)$.
%     \item When we proved \emph{strong authentication}, it failed due
%       to the lack of freshness property generating The Dolev and Yao
%       attacker (of AVISPA tool) a replay attack. However, this attack
%       is already not possible when we consider that $K'_{CS}$ is
%       fresh, which was obtained in an immediately previous step. Even
%       though, we could include timestamps in all steps, however, that
%       is not necessary. 
% \end{itemize}

\subsubsection{Client-Miner Authentication}
\label{ssec:SecAuth}


The authentication process trusts in the users' \textit{username} 
(email) and \textit{password}, so, it is responsibility for the user 
to ensure his secrets. However, it is well known that any system 
cannot trust only in a user and password control access, because of that, 
it is important not only to encrypt the sensible information, but also
to implement more security mechanisms (like authentication procedure) to 
avoid possible attacks.

The goal of this stage is to establish authentication between a client and 
a miner after having a secure tunneling. Note that, public and private key
of the user are not used for the authentication process, it is only used to
do transactions. Table~\ref{table:AuthProtocol} 
describes the authentication part and the explanation is as follows:

\begin{itemize}
  \item Both the client and the miner store all knowledge accumulated
    of previous steps. It is considered to have established a secure tunneling 
    with the miner using the TLS protocol and to have obtained the register
    procedure. So, the initial knowledge of the client consists in to know 
    the session key $K_{\Client\Server}$, his password $P$ and the secret $X$. 
    On the other hand, the miner knows the same session key, the client public and
    private key, personal information of the client, secret $X$ and the hashed 
    $Hash(P,X)$. The following steps are proposed:
    \begin{itemize}
    \item \textbf{First step:} the client sends a challenge to the
      miner $N_\Client$. The miner can map $@$ with $\Client$ and build $hash(@\lnk\Client\Server\lnk N_\Client)$. This message is ciphered with $K_{\Client\Server}$ because the session key has been destined for them. 
    \item \textbf{Second step:} the server responds to the client with a new challenge,
      firstly includes the client name $C$ (which is mapped with $@$) and a hashed message
      $Y$ composed by $N_{\Client}$ and $X$ both messages must be known by the client. Again,
      all ciphered with $K_{\Client\Server}$.
    \item \textbf{Third step:} Once the client has accepted the previous step, it means that
      he has deduced all previous messages and he has authenticated to the miner, so, the 
      client sends back his username $@$, the challenge response $N_\Client$ as a freshness
      property and his password concatenated with $P$, $X$ and $N_{\Client}$.
  \end{itemize}
\item Finally, after having run the protocol it is expected to obtain strong authentication
    between the client and the miner. 
\end{itemize}
 
\begin{table}[htb]
\footnotesize
\begin{center}
\caption{Authentication Protocol}
\label{table:AuthProtocol}
\begin{tabular}{|l|l|l|}
\hline
{\bf Process}                                                                     \\\hline\hline
            Initial Knowledge                                                     \\
            $\Client:P\lnk X \lnk K_{\client\serverW}$                               \\
            \hspace{5mm}                                                 \\ 
            $\Server:Hash(P,X)\lnk X \lnk[@::Pd] \lnk K_{\client\serverW}$ \\\hline \hline
            Authentication                                                         \\
            1.-$\Client\rightarrow \Server: \fat{\Server \lnk @ \lnk N_{\Client} \lnk 
                                              Hash(@\lnk \Client \lnk \Server \lnk N_\Client)}_{K_{\client\serverW}}$  \\ 
            2.-$\Server\rightarrow \Client: \fat{\Client \lnk Hash(N_{\Client},X)}_{K_{\client\serverW}}$ \\ 
            \hspace{5mm} $Y=hash(N_{\Client},hash(P,X))$                          \\              
            3.-$\Client\rightarrow \Server: \fat{@\lnk N_{\Client}\lnk Y}_{K_{\client\serverW}}$            \\ \\ \hline  \hline
            \textbf{Security Properties}                        \\
            \hspace{5mm} $\Client$ authenticates $\Server$ on $N_{\Client}$        \\
            \hspace{5mm} $\Server$ authenticates $\Client$ on $Y$  \\ 
            \hspace{5mm} Secrecy of $P$ known by $[\Client]$ \\\hline \hline
\end{tabular}
\end{center}
\end{table}
\normalsize




\subsubsection{Get Services with authentication}
\label{ssec:getServAuth}
Some services are available only after the authentication process, for example, to know:
\begin{itemize}
    \item A history of owners
    \item How many times the vehicle has been arrested 
    \item The number of penalties
    \item A history of taxes.
\end{itemize}

See Table~\ref{table:servicesPermits} to identify what services require authentication.
Table~\ref{table:ProtGetServicesAuth} describes in Alice and Bob notation how to get services 
requiring an authentication process, the explanation is similar to that in 
Section~\ref{ssec:getServNoAuth}, next we explain the differences. 

As you can see, the difference with respects to 
Table~\ref{table:ProtGetServicesNoAuth}, where authentication is not required, is that in steps
1 and 2, the hashed message $Y$ is include.

Here, after step 1, the miner verifies if such a service can be consumed (according permissions, see
Table~\ref{table:servicesPermits}) and then, returns the answer $d$; in a ciphered way as above was 
explained.






\begin{table}[htb]
\footnotesize
\begin{center}
\caption{Get Service Protocol with authentication}
\label{table:ProtGetServicesAuth}
\begin{tabular}{|l|}
\hline
           Initial Knowledge                                                             \\
            $C:C\lnk QR \lnk  K_{\client\serverW} \lnk Y$                                    \\
            $\Server: \Server\lnk K_{\client\serverW} \lnk Y$    \\ \hline \hline 
           1.-$\Client\rightarrow \Server: \fat{C \lnk QR \lnk n \lnk Y}_{K_{\client\serverW}}$          \\ 
           \textbf{Get a service}                                                                        \\
           \hspace{5mm} $d=getService(QR,n)$                                  \\  
           2.-$\Server\rightarrow \Client: \fat{C \lnk d \lnk Y}_{K_{\client\serverW}}$       \\ \\   \hline \hline
\end{tabular}
\end{center}
\end{table}
\normalsize





