\subsection{Data provenance}
Cloud computing allows the existence of digital representations of phyisical world objects in a permanent way, independent of a single copy existing on a specific location. This digital representation can contain information of the whole life cycle of the object, its accountability and historical changes. Specifically among cloud technologies, blockchain allows for the continuous existence of the digital representations introducing security aspects like traceability and resistance to unwanted modifications. Implementations have been made to preserve digital representations \cite{provchain}. Kim and Laskowski\cite{ontology} show an approach using Ethereum technology to propose ontologies that can represent knowledge provenance. The blockchain approach is having a great momentum in the data provenance medium for it theoretical strengths in security and data protection. Neisse et al. \cite{europe} give this a push by proposing enforcement of smart contracts to fulfill legal obligations of personal data over data controllers and processsors in Europe using blockchain. Blockchain provenance is not limited to storage information. Smart contracts inside the transactions allow for programmable interactive environments \cite{ramachandran} where the blockchain will be enriched with valid data and binding transactions every time something interesting happens to the physical part of the object.